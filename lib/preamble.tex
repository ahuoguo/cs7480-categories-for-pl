\usepackage{tikz,tikz-cd,relsize,hyperref}
\usepackage[nameinlink,capitalise]{cleveref} \usepackage{enumitem}
\usepackage{mathtools,subcaption}
\usepackage{amsmath,amssymb,stmaryrd,mathpartir,listings,xcolor}
\usepackage[utf8]{inputenc} % for hiragana yo
\usepackage{mdframed}

\usetikzlibrary{positioning,arrows}
        
\mdfdefinestyle{theoremstyle}{%
  linecolor=black,linewidth=1pt,%
  frametitlerule=true,%
  frametitlebackgroundcolor=gray!20,
  innertopmargin=\topskip,
}
\mdtheorem[style=theoremstyle]{definition}{Definition}

\definecolor{codegreen}{rgb}{0,0.6,0}
\definecolor{codegray}{rgb}{0.5,0.5,0.5}
\definecolor{codepurple}{rgb}{0.58,0,0.82}
\definecolor{backcolour}{rgb}{0.95,0.95,0.92}

\lstdefinestyle{mystyle}{
    backgroundcolor=\color{backcolour},   
    commentstyle=\color{codegreen},
    keywordstyle=\color{magenta},
    numberstyle=\tiny\color{codegray},
    stringstyle=\color{codepurple},
    basicstyle=\ttfamily\footnotesize,
    breakatwhitespace=false,         
    breaklines=true,                 
    captionpos=b,                    
    keepspaces=true,                 
    numbers=left,                    
    numbersep=5pt,                  
    showspaces=false,                
    showstringspaces=false,
    showtabs=false,                  
    tabsize=2
}

\lstset{style=mystyle}


% This order is important: \newtheorem needs to be after importing \cleveref
% https://stackoverflow.com/questions/6499504/cleveref-fails-for-theorem-environments-sharing-the-same-counter
\theoremstyle{definition}
\newtheorem{theorem}{Theorem}[section]
\newtheorem{corollary}[theorem]{Corollary}
\newtheorem{lemma}[theorem]{Lemma}
\newtheorem{hope}[theorem]{Hope}
\newtheorem{belief}[theorem]{Belief}
\newtheorem{construction}[theorem]{Construction}
\newtheorem{intuition}[theorem]{Intuition}
% \newtheorem{definition}[theorem]{Definition}
\newtheorem{notation}[theorem]{Notation}
\newtheorem{example}[theorem]{Example}
\newtheorem{warning}[theorem]{Warning}
\newtheorem{claim}[theorem]{Claim}
\newtheorem{procedure}[theorem]{Procedure}
\newtheorem{note}[theorem]{Note}
\newtheorem{question}[theorem]{Question}
\newtheorem{fact}[theorem]{Fact}
\newtheorem{assumption}[theorem]{{\color{red}Assumption}}
\newtheorem{metatheorem}[theorem]{Metatheorem}
\newtheorem{proposition}[theorem]{Proposition}
\newtheorem{remark}[theorem]{Remark}
\newtheorem{problem}{Problem}
\newtheorem{problem-part}{Part}[problem]
\newenvironment{answer}{\renewcommand{\proofname}{Answer}\begin{proof}}{\end{proof}}
\crefname{theorem}{Theorem}{Theorems}
\crefname{construction}{Construction}{Constructions}
\crefname{corollary}{Corollary}{Corollaries}
\crefname{lemma}{Lemma}{Lemmas}
\crefname{hope}{Hope}{Hopes}
\crefname{belief}{Belief}{Beliefs}
\crefname{construction}{Construction}{Constructions}
\crefname{intuition}{Intuition}{Intuitions}
\crefname{definition}{Definition}{Definitions}
\crefname{notation}{Notation}{Notations}
\crefname{example}{Example}{Examples}
\crefname{warning}{Warning}{Warnings}
\crefname{claim}{Claim}{Claims}
\crefname{procedure}{Procedure}{Procedures}
\crefname{note}{Note}{Notes}
\crefname{question}{Question}{Questions}
\crefname{fact}{Fact}{Facts}
\crefname{assumption}{Assumption}{Assumptions}
\crefname{metatheorem}{Metatheorem}{Metatheorems}
\crefname{proposition}{Proposition}{Propositions}
\crefname{metatheorem}{Metatheorem}{Metatheorems}
\crefname{problem}{Problem}{Problems}
\crefname{problem-part}{Part}{Parts}

\DeclareFontFamily{U}{min}{}
\DeclareFontShape{U}{min}{m}{n}{<-> udmj30}{}
\newcommand\yo{\!\text{\usefont{U}{min}{m}{n}\symbol{'207}}\!}

\newcommand\subtext[1]{{\color{lightgray}#1}}

\newcommand\Set{\mathsf{Set}}
\newcommand\FinSet{\mathsf{FinSet}}
\newcommand\Cat{\mathsf{Cat}}
\newcommand\Vect{\mathsf{Vect}}
\newcommand\Group{\mathsf{Group}}
\newcommand\ofty{\,{:}\,}
\newcommand\bool{\mathrm{bool}}
\newcommand\true{\mathrm{true}}
\newcommand\false{\mathrm{false}}
\newcommand\colim{\operatorname{colim}}
\newcommand\copi{\text{\rotatebox[origin=c]{180}{$\pi$}}}
\newcommand\adjunct{\mathop{\text{\reflectbox{$\rightleftarrows$}}}}
\newcommand\img{\operatorname{im}}
\newcommand\Sh{\operatorname{Sh}}
\newcommand\psh{\operatorname{PSh}}
\newcommand\Aut{\operatorname{Aut}}
\newcommand{\ama}{\operatorname{\mathsf{amalg}}}
\newcommand{\dom}{\operatorname{\mathsf{dom}}}
\newcommand{\cod}{\operatorname{\mathsf{cod}}}
\newcommand{\idt}{\mathsf{id}}
\newcommand\todo{{\color{red}\bf todo}}
\newcommand\Sub{\operatorname{Sub}}
\newcommand\Hom{\operatorname{Hom}}
\newcommand\id{\operatorname{id}}
\newcommand\dnclose{\operatorname{\downarrow}}
\newcommand\genelt[2]{\mathsf{Elt}_{#2}(#1)}
\newcommand\calA{\mathcal A}
\newcommand\calB{\mathcal B}
\newcommand\calC{\mathcal C}
\newcommand\calD{\mathcal D}
\newcommand\calE{\mathcal E}
\newcommand\calF{\mathcal F}
\newcommand\calG{\mathcal G}
\newcommand\calH{\mathcal H}
\newcommand\calI{\mathcal I}
\newcommand\calJ{\mathcal J}
\newcommand\calK{\mathcal K}
\newcommand\calL{\mathcal L}
\newcommand\calM{\mathcal M}
\newcommand\calN{\mathcal N}
\newcommand\calO{\mathcal O}
\newcommand\calP{\mathcal P}
\newcommand\calQ{\mathcal Q}
\newcommand\calR{\mathcal R}
\newcommand\calS{\mathcal S}
\newcommand\calT{\mathcal T}
\newcommand\calU{\mathcal U}
\newcommand\calV{\mathcal V}
\newcommand\mor[1]{\xrightarrow{#1}}
\newcommand\llbr[1]{\left\llbracket #1\right\rrbracket}
\newcommand\ite[3]{\mathrm{if}~#1~\mathrm{then}~#2~\mathrm{else}~#3}
\newcommand\R{\mathbb R}
\newcommand\N{\mathbb N}
\newcommand\Q{\mathbb Q}
\newcommand\Z{\mathbb Z}
\newcommand\Ber{\operatorname{Ber}}
\newcommand\Ob{\operatorname{Ob}}
\newcommand\angled[1]{\langle #1\rangle}
\newcommand\frakA{\mathfrak A}
\newcommand\frakB{\mathfrak B}
\newcommand\frakC{\mathfrak C}
\newcommand\frakD{\mathfrak D}
\newcommand\frakE{\mathfrak E}
\newcommand\frakF{\mathfrak F}
\newcommand\frakG{\mathfrak G}
\newcommand\frakH{\mathfrak H}
\newcommand\frakM{\mathfrak M}
\newcommand\frakN{\mathfrak N}
\newcommand\frakO{\mathfrak O}
\newcommand\frakP{\mathfrak P}
\newcommand\frakQ{\mathfrak Q}
\newcommand\frakU{\mathfrak U}
\newcommand\frakV{\mathfrak V}
\newcommand\stepsto{\rightarrow}

\newcommand\plkwcolor[1]{{\color{blue}#1}}
\newcommand\pllitcolor[1]{{\color{teal}#1}}
\newcommand\plfont[1]{\mathsf{#1}}
\newcommand\dbracket[1]{\left\llbracket #1 \right\rrbracket}
\newcommand\sem[1]{{\color{blue} #1}}
\newcommand\plkw[1]{\plfont{\plkwcolor{#1}}}
\newcommand\pllit[1]{\mathsf{\pllitcolor{#1}}}
\newcommand\pllet[3]{\plkw{let}~#1~\plkw{be}~#2~\plkw{in}~#3}
\newcommand\plret[1]{\plkw{ret}~#1}
\newcommand\plproj[2]{\plkw{proj}_{\plkwcolor{#1}}~#2}
\newcommand\plinj[2]{\plkw{inj}_{\plkwcolor{#1}}~#2}

\newcommand\plUnit{\plkw{Unit}}
\newcommand\plunit{{\plkwcolor{\angled{}}}}

\newcommand\pltimes{\mathbin{\plkwcolor\times}}
\newcommand\plpair[2]{\plkwcolor{\angled{{\normalcolor #1\plkwcolor{,}\, #2}}}}
\newcommand\plfst[1]{\plkw{fst}\,#1}
\newcommand\plsnd[1]{\plkw{snd}\,#1}
\newcommand\plflip[0]{\plkw{flip}}
\newcommand\pltrue[0]{\plkw{true}}
\newcommand\plfalse[0]{\plkw{false}}
\newcommand\plBool[0]{\plkw{Bool}}

\newcommand\plto{\mathbin{\plkwcolor\to}}
\newcommand\pllam[2]{\plkwcolor{\lambda}\,#1\plkwcolor{.}\,#2}
\newcommand\plapp[2]{#1\,#2}

\newcommand\plVoid{\plkw{Void}}
\newcommand\plunreachable[1]{\plkw{unreachable}\,#1}

\newcommand\plplus{\mathbin{\plkwcolor+}}
\newcommand\plinl[1]{\plkw{inl}\,#1}
\newcommand\plinr[1]{\plkw{inr}\,#1}
\newcommand\plcase[5]
{\plkw{case}\plkwcolor{(}#1\plkwcolor{,}~
  #2\plkwcolor{.}\,#3\plkwcolor{,}~
  #4\plkwcolor{.}\,#5)}

% https://tikzcd.yichuanshen.de/#N4Igdg9gJgpgziAXAbVABwnAlgFyxMJZABgBpiBdUkANwEMAbAVxiRAZAF9T1Nd9CKMgEYqtRizYMA5FzEwoAc3hFQAMwBOEALZIyIHBCTDq9Zq0Qg1IagzoAjGAwAKfPATYasigBY45nEA
% 1
% |
% | 2
% v
% 3
\newcommand\vertarrow[3]{
  \begin{tikzcd}[ampersand replacement=\&]
    {#1} \arrow[d, "{#2}"'] \\
    {#3}               
  \end{tikzcd}
}

%      2
%   1 --> 3
% 4 |     | 5
%   v     v
%   6 --> 8
%      7
\newcommand\commsquare[8]{
\begin{tikzcd}[ampersand replacement=\&]
{#1} \arrow[d, "{#4}"'] \arrow[r, "{#2}"] \& {#3} \arrow[d, "{#5}"] \\
{#6} \arrow[r, "{#7}"']                \& {#8}
\end{tikzcd}
}

%      2
%   1 --> 3
% 4 | _|  | 5
%   v     v
%   6 --> 8
%      7
\newcommand\pullback[8]{
\begin{tikzcd}[ampersand replacement=\&]
{#1} \arrow[d, "{#4}"'] \arrow[r, "{#2}"]
\arrow[dr, phantom, "\lrcorner", very near start]
\& {#3} \arrow[d, "{#5}"] \\
{#6} \arrow[r, "{#7}"'] \& {#8}
\end{tikzcd}
}

% takes the arguments for \pullback first, and then takes arguments for the cone, then mediating map
\newcommand\pullbackump[3]{%
  \def\tempa{#1}%
  \def\tempb{#2}%
  \def\tempc{#3}%
  \pullbackumprest
}
\newcommand\pullbackumprest[9]{
% https://tikzcd.yichuanshen.de/#N4Igdg9gJgpgziAXAbVABwnAlgFyxMJZARgBpiBdUkANwEMAbAVxiRDpAF9T1Nd9CKAEzkqtRizYAjLjxAZseAkTJCx9Zq0QgAxrN6KBREWuobJ2qPvl8lg5AAZSD9RK0hWnMTCgBzeESgAGYAThAAtkgALNQ4EEgiIFIwYFaIALQAzE7immwAFiDUDHTJDAAKtkbaIVi++TjWoRHRsfGIZEkpadlmbmwA1k1hkYg5cUid5u5oQcMtHW1ImX152r7zo4kTiCu5FiBzxaUwFVXKNXUNm0jj7YnTbGgbx2WVhhcgtfWN3MEjrRAOxyJTe50EXyujVWByYXAonCAA
\begin{tikzcd}[ampersand replacement=\&]
  #6 \arrow[rdd, "#7"', bend right] \arrow[rrd, "#8", bend left] \arrow[rd, "#9"'] \&                                    \&                  \\
     \& \tempa \arrow[r, "\tempb"] \arrow[d, "#1"'] 
        \arrow[dr, phantom, "\lrcorner", very near start]
     \& \tempc \arrow[d, "#2"] \\
     \& #3 \arrow[r, "#4"']                  \& #5               
  \end{tikzcd}
}

%    2     4
% 1 --> 3 --> 6
%         -->
%          5 
\newcommand\weakequalizer[6]{
  \begin{tikzcd}[ampersand replacement=\&]
      #1 \arrow[r, "#2"'] 
      \& #3 \arrow[r, "#5"', shift right=1] \arrow[r, "#4", shift left=1]
      \& #6
  \end{tikzcd}
}

%    2     4
% 1 --> 3 --> 6
%         -->
%          5 
% hookrightarrow on 2
\newcommand\equalizer[6]{
  \begin{tikzcd}[ampersand replacement=\&]
      #1 \arrow[r, "#2"', hook] 
      \& #3 \arrow[r, "#5"', shift right=1] \arrow[r, "#4", shift left=1]
      \& #6
  \end{tikzcd}
}

%    2     5
% 1 --> 4 --> 6
%   --> 
%    3  
\newcommand\weakcoequalizer[6]{
  \begin{tikzcd}[ampersand replacement=\&]
      #1 \arrow[r, "#3"', shift right=1] \arrow[r, "#2", shift left=1]
      \& #4 \arrow[r, "#5"'] 
      \& #6
  \end{tikzcd}
}

%    2         4
% 1 --> 3 ----------> 6
%         ---------->
%              5
\newcommand\longequalizer[6]{
  \begin{tikzcd}[ampersand replacement=\&]
      #1 \arrow[r, "#2"', hook] 
      \& #3 \arrow[rr, "#5"', shift right=1] \arrow[rr, "#4", shift left=1]
      \& {} \& #6
  \end{tikzcd}
}

%    2
% 1 <-- 4
%   -->
%    3
\newcommand\adjunction[4]{
  {#1}~~\underset{\mathlarger{\underset {#3}\longrightarrow}}{\overset{\mathlarger{\overset {#2}\longleftarrow}}{\mathsmaller{\mathsmaller\bot}}}~~{#4}
}

%    2
% 1 ---> 3
%   \   /
%  4 \ / 5     (diagonal arrow on the right dashed)
%     6
\newcommand\uniqext[6]{
  \begin{tikzcd}[ampersand replacement = \&]
  #1 \arrow[rd, "#4"'] \arrow[rr, "#2"] \&   \& #3 \arrow[ld, "#5", dashed] \\
  \& #6 \&
  \end{tikzcd}
}

%   2      4
% 1 --> 3 --> 5
% |     |     |
% |6    |7    |8
% v     v     v
% 9 --> 11 -> 13
%   10     12
%
% https://tikzcd.yichuanshen.de/#N4Igdg9gJgpgziAXAbVABwnAlgFyxMJZABgBpiBdUkANwEMAbAVxiRDpAF9T1Nd9CKAIzkqtRizYAjLjxAZseAkQBMo6vWatEIAMazeigUTJCxmyTqgH5fJYOQizGidpCtuh-spRrn4rTYAMy4xGCgAc3giUCCAJwgAWyQyEBwIJBEAyxAAMRt4pMzqdKQ1bLcAcQKE5MRU0sQAZhdAnQAJGqLELMaAFlacgEkuuvLGgFZBtwApUaQWtIzEAYq2AGkQagY6KRgGAAU7Yx04rAiACxx5lZLlqbWdABktkB29w+OfEDPL684KJwgA
\newcommand\commrect[6]{%
  \def\tempa{#1}%
  \def\tempb{#2}%
  \def\tempc{#3}%
  \def\tempd{#4}%
  \def\tempe{#5}%
  \def\tempf{#6}%
  \commrectrest
}
\newcommand\commrectrest[7]{%
\begin{tikzcd}[ampersand replacement = \&]
\tempa \arrow[r, "\tempb"] \arrow[d, "\tempf"] \& \tempc \arrow[r, "\tempd"] \arrow[d, "#1"] \& \tempe \arrow[d, "#2"] \\
#3 \arrow[r, "#4"']               \& #5 \arrow[r, "#6"']               \& #7
\end{tikzcd}
}

%     2
%   ---->    5
% 1       4 --->> 6
%   ---->
%     3
\newcommand\coequalizer[6]{
% https://tikzcd.yichuanshen.de/#N4Igdg9gJgpgziAXAbVABwnAlgFyxMJZABgBpiBdUkANwEMAbAVxiRAGIBGEAX1PUy58hFJ3JVajFm3YAWXvxAZseAkQBM46vWatEHAGy8JMKAHN4RUADMAThAC2SMiBwQkYkHAAWWazg9tKT0OAGYQagY6ACMYBgAFQVUREFssM28Avht7J0QXN0CvX38kAFpPHWl9dnUFHMciwsRNSV0ZAFYIkG8YOig2HAB3CF7+hB4KHiA
\begin{tikzcd}[ampersand replacement=\&]
  {#1} \arrow[r, "{#3}"', shift right] \arrow[r, "{#2}", shift left] \& {#4} \arrow[r, "{#5}", two heads] \& {#6}
  \end{tikzcd}
}

%       2
%  1  ---->>  3
%  |          |
% 4|          |5
%  v          v
%  6 hook---> 8
%        7
\newcommand\strongepiunfilled[8]{
% https://tikzcd.yichuanshen.de/#N4Igdg9gJgpgziAXAbVABwnAlgFyxMJZABgBpiBdUkANwEMAbAVxiRAGIBGEAX1PUy58hFJ3JVajFm3YBmXvxAZseAkTKcJ9Zq0QcAbAoErhRMZurbpe9gA5eEmFADm8IqABmAJwgBbJGQgOBBIYpI6MgBMINQAFjB0UGw4AO4Q8YkIfJ4+-oiBwUiRllK6HAAsMSAMdABGMAwACoKqIiBeWM6xOEYg3n6h1IWIsiURNgCsvf15xUEhI0N0WAxssRAQANZVVmXsAOxVNfVNLaZ6HV09PBQ8QA
\begin{tikzcd}[ampersand replacement=\&]
  {#1} \arrow[r, "{#2}", two heads] \arrow[d, "{#4}"'] \& {#3} \arrow[d, "{#5}"] \\
  {#6} \arrow[r, "{#7}"', hook]                      \& {#8}                
  \end{tikzcd}
}

%       2
%  1  ---->>   3
%  |     /     |
% 4|  9/dashed |5
%  v v         v
%  6 hook---> 8
%        7
\newcommand\strongepi[9]{
% https://tikzcd.yichuanshen.de/#N4Igdg9gJgpgziAXAbVABwnAlgFyxMJZABgBpiBdUkANwEMAbAVxiRAGIBGEAX1PUy58hFJ3JVajFm3YBmXvxAZseAkTKcJ9Zq0QcAbAoErhRMZurbpe9gA5eEmFADm8IqABmAJwgBbJGQgOBBIYpI6MgBMINQAFjB0UGw4AO4Q8YkIfJ4+-oiBwUiRllK6HAAsMSAMdABGMAwACoKqIiBeWM6xOEYg3n6h1IWIsiURNgCsvf15xUEhI0N0WAxssRAQANZVVmXsAOxVNfVNLaZ6HV092X25g-NFY9YcAJxHWGBlUHRw8Uk8FB4QA
\begin{tikzcd}[ampersand replacement=\&]
  {#1} \arrow[r, "{#2}", two heads] \arrow[d, "{#4}"'] \& {#3} \arrow[d, "{#5}"] \arrow[ld, "{#9}", dashed] \\
  {#6} \arrow[r, "{#7}"', hook]                      \& {#8}                                         
\end{tikzcd}
}

%       3
%      ^ \hook
%   2 /   \4
%    /     v
%  1 ------>> 6
%       5
\newcommand\extremalepi[6]{
  % https://tikzcd.yichuanshen.de/#N4Igdg9gJgpgziAXAbVABwnAlgFyxMJZABgBoBGAXVJADcBDAGwFcYkRgBicgXxB9LpMufIRTlSxanSat2XAMx8BQ7HgJEATBWkMWbRB04A2ZdJhQA5vCKgAZgCcIAWyRkQOCEgkz98zprKgiCOLm40nkjaIAAWMPRQ7DgA7hBxCQg0enKGXACsfDSM9ABGMIwACsLqYiAOWJYxOPzBoa6IPpGI0Tj0WIzsMRAQANYgWbIGRgAsZjxAA
\begin{tikzcd}[ampersand replacement=\&]
  \& {#3} \arrow[rd, "{#4}", hook] \&      \\
{#1} \arrow[ru, "{#2}"] \arrow[rr, "{#5}"', two heads] \&                               \& {#6}
\end{tikzcd}
}


